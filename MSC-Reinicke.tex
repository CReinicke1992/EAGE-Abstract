\documentclass{madrid15WS}

\usepackage{amssymb}
\usepackage{amsmath}
%\usepackage[FIGTOPCAP, hang, nooneline]{subfig}
\usepackage{subcaption}
\usepackage{placeins}
\usepackage{natbib}
\usepackage{siunitx}
%\usetikzlibrary{matrix}
\usepackage{todonotes}
\usepackage{caption}
\usepackage{verbatim}
\usepackage{rotating}


\begin{document}

\section{Introduction}

Blending is a recent seismic acquisition design, which allows seismic sources to interfere. Currently, in marine seismic acquisition sources are blended in the inline direction, i.e. the data are blended in 2D (inline direction and time). In a pre-processing step the blended data must be deblended (separated) as if they were acquired in a conventional way. 

In this abstract I propose to blend marine sources in the crossline direction. Thus, in combination with the movement of the vessel the data are blended in 3D (crossline direction, inline direction and time). 3D blending allows to design many new acquisition configurations. Figure \ref{fig:Intro-Configs} shows a conventional acquisition design compared to one possible 3D blended acquisition design. In the following I will refer to the acquisition design in Figure \ref{fig:Intro-Config-Xline-Blended} as wide crossline source array. Both acquisition designs in Figure \ref{fig:Intro-Configs} acquire the same area. However, the wide crossline source array requires less equipment and benefits from a more symmetric distribution of sources and receivers.

Existing deblending methods are designed for 2D blended data. Therefore, I will introduce a 3D deblending method. 

\begin{figure}[h!]
	\centering
	\begin{subfigure}[t]{0.3\textwidth}
		\centering
		\includegraphics[width = \textwidth]{Plots/Config-Conventional}
		\caption{Conventional acquisition \\design}
		\label{fig:Intro-Config-Conventional}
	\end{subfigure}
	\qquad \qquad 
	\centering
	\begin{subfigure}[t]{0.3\textwidth}
		\centering
		\includegraphics[width = \textwidth]{Plots/Config-Xline-Blended}
		\caption{Wide crossline source array}
		\label{fig:Intro-Config-Xline-Blended}
	\end{subfigure}
	
	\caption{}
	\label{fig:Intro-Configs}
	
\end{figure}


\section{Method and/or Theory}

In 3D acquisition the sources and receivers are distributed on a 2D surface. Thus, their locations are defined by their inline and crossline positions, (x, y). Each data point which is measured by a source receiver pair at a specific time is therefore described by five coordinates, time t, receiver inline and crossline position (xr, yr), and source inline and crossline position (xs, ys).


\section{Examples (Optional)}

This is the first sentence of the example section.

\section{Results (Optional)}

This is the first sentence of the result section.

% \begin{figure}[!htb]
%   \centering
%   \includegraphics[width=0.6\textwidth]{....eps}
%   \caption{...}
% \end{figure}

\section{Conclusions}

This is the first sentence of the conclusions.

\section{Acknowledgements (Optional)}

This is the first sentence of the acknowledgements.

% \begin{thebibliography}{6pt}
%   \bibitem[{<reference>}]{<cite>} ...
% \end{thebibliography}
%
% or
%
% \bibliography{...}

% Biblio
\bibliographystyle{apalike}
\bibliography{my_bib} % in MyBib.bib you add all your reference information, following the correct format. Sometimes, the bib file needs to be built several times, as well as the main file, before all references occur correctly in your PDF. 


\end{document}