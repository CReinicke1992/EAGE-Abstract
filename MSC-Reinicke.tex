\documentclass{madrid15WS}

\usepackage{amssymb}
\usepackage{amsmath}
%\usepackage[FIGTOPCAP, hang, nooneline]{subfig}
\usepackage{subcaption}
\usepackage{placeins}
\usepackage{natbib}
\usepackage{siunitx}
%\usetikzlibrary{matrix}
\usepackage{todonotes}
\usepackage{caption}
\usepackage{verbatim}
\usepackage{rotating}


\begin{document}

\section{Introduction}

Blending is a recent seismic acquisition design, which allows seismic sources to interfere. Currently, in marine seismic acquisition sources are blended in the inline direction, i.e. the data are blended in 2D (inline direction and time). A pre-processing step is required to deblend (separate) the blended data as if they were acquired in a conventional way. 

In this abstract I propose to blend marine sources in the crossline direction. Thus, in combination with the movement of the vessel the data are blended in 3D (crossline direction, inline direction and time). 3D blending allows to design many new acquisition configurations. Figure \ref{fig:Intro-Configs} shows a conventional acquisition design compared to one possible 3D blended acquisition design. In the following I will refer to the acquisition design in Figure \ref{fig:Intro-Config-Xline-Blended} as wide crossline source array. Both acquisition designs in Figure \ref{fig:Intro-Configs} acquire the same area. However, the wide crossline source array requires less equipment and benefits from a more symmetric distribution of sources and receivers.

Existing deblending methods are designed for 2D blended data. Therefore, I will introduce a 3D deblending method. 

\begin{figure}[h!]
	\centering
	\begin{subfigure}[t]{0.3\textwidth}
		\centering
		\includegraphics[width = \textwidth]{Plots/Config-Conventional}
		\caption{Conventional acquisition \\design}
		\label{fig:Intro-Config-Conventional}
	\end{subfigure}
	\qquad \qquad 
	\centering
	\begin{subfigure}[t]{0.3\textwidth}
		\centering
		\includegraphics[width = \textwidth]{Plots/Config-Xline-Blended}
		\caption{Wide crossline source array}
		\label{fig:Intro-Config-Xline-Blended}
	\end{subfigure}
	
	\caption{}
	\label{fig:Intro-Configs}
	
\end{figure}


\section{Method and/or Theory}

The following method is based on my master thesis \citep{myself}.

\subsection{Data sorting}

In 3D acquisition the sources and receivers are distributed on a 2D surface. Thus, their locations are defined by their inline and crossline positions, ($x$, $y$). Each data point which is measured by a source receiver pair at a specific time is therefore described by five coordinates, time $t$, receiver inline and crossline position ($x_r$, $y_r$), and source inline and crossline position ($x_s$, $y_s$).

The 5D data ”cube” is reorganized in a 2D data matrix according to \citet{Delphi-Format} (see Figure \ref{fig:DelphiFormat}). For this data sorting a 1D Fourier transform with respect to time is performed and a 4D frequency ”slice” is selected.

The 4D ”slice” is sorted in a 2D data matrix, $\mathbf{P}$, with as many rows as receivers and as many columns as shots. The total number of shots is obtained by multiplying the number of shots fired in each crossline and the number of shots fired in each inline. The total number of receivers is obtained likewise. Assume there are $Ns_x$ shots per crossline. The shots of the first crossline are assigned to the first $Ns_x$ columns of the data matrix, the shots of the second crossline are assigned to the next $Ns_x$ columns of the data matrix, etc. The receivers are sorted in the rows of the data matrix analogously.

One row in the data matrix, $\mathbf{P}$, in Figure \ref{fig:DelphiFormat} represents a 3D common-receiver gather. The data of this 3D common-receiver gather are shown in Figure \ref{fig:CRG_3D-view} in a 3D-view where the coordinates, $x$ and $y$, indicate the inline and crossline shot position respectively. For the described data sorting individual crossline slices are extracted from this data cube and assembled next to each other in a data matrix as shown in Figure \ref{fig:CRG_2D-view}. This view will be referred to as 3D CRG 2D-view. Each hyperbolic event in Figure \ref{fig:CRG_2D-view} refers to the response of the shots of one crossline.

\begin{figure}[h!]
	
	\centering
	\begin{subfigure}[t]{0.4\textwidth}
		\centering
		\includegraphics[width = \textwidth]{Plots/DelphiFormat-v3}
		\caption{Data matrix $\mathbf{P}$}
		\label{fig:DelphiFormat}
	\end{subfigure}
	\qquad  
	\centering
	\begin{subfigure}[t]{0.3\textwidth}
		\centering
		\includegraphics[width = \textwidth]{Plots/data3d}
		\caption{CRG 3D-view}
		\label{fig:CRG_3D-view}
	\end{subfigure}
	
	\centering
	\begin{subfigure}[t]{0.8\textwidth}
		\centering
		\includegraphics[width = \textwidth]{Plots/p_Delphi}
		\caption{CRG 2D-view}
		\label{fig:CRG_2D-view}
	\end{subfigure}
	
	\caption{(a) Illustration of the data matrix, $\mathbf{P}$, for 3D data \citep{Delphi-Format}. (b) 3D- view of a 3D common-receiver gather. (c) 2D-view of a 3D common-receiver gather.}
	\label{fig:DataSorting}
\end{figure}


\subsection{Deblending strategy}

The presented deblending strategy is similar to the 2D deblending method of \citet{Mahdad-Deblending-Method}: The blended data are used to build a pseudo-deblended dataset. \citet{Mahdad-Deblending-Method} shows that the energy of the blended sources becomes incoherent in a common-receiver gather of the pseudo-deblended data. The incoherent energy generated by the blended sources is referred to as blending noise. Therefore, I will use a coherency constraint in the $f$-$k_x$-$k_y$ domain, and a sparsity constraint in the $x$-$t$ domain to remove the blending noise from the pseudo-deblended data. Figure \ref{fig:3D-CRG-Pps} shows an example of a pseudo-deblended 3D common-receiver gather in 2D-view. 

\begin{figure}[h!]
	\centering
	\begin{subfigure}[t]{0.8\textwidth}
	\includegraphics[width=\textwidth]{Plots/3dDeblending/Pseudo-Deblendedv5_xt_100}
	\caption{}
	\label{fig:3D-CRG-Pps}
	\end{subfigure}
	
	\centering
	\begin{subfigure}[t]{0.8\textwidth}
	\includegraphics[width=\textwidth]{Plots/3dDeblending/Deblendedv5_xt_100}
	\caption{}
	\label{fig:3D-CRG-Pdebl}
	\end{subfigure}
	
	\centering
	\begin{subfigure}[t]{0.8\textwidth}
	\includegraphics[width=\textwidth]{Plots/3dDeblending/Unblended_Delphi_zoom-font14}
	\caption{}
	\label{fig:3D-CRG-Punbl}
	\end{subfigure}
	
	\caption{Illustration of 3D common-receiver gathers in 2D-view for (a) pseudo-deblended data, (b) deblended data, and (c) unblended data.}
	\label{fig:3D-CRG-2D-view}
\end{figure}

\subsection{3D $f$-$k_x$-$k_y$ filter}

A 3D common-receiver gather has two spatial directions ($x$,$y$), i.e. the frequency-wavenumber domain is 3D ($f$-$k_x$-$k_y$).

For this purpose one considers a 3D common-receiver gather, $\mathbf{p}_{ps}(t, x_s, y_s)$, and brings it to the $f$-$k_x$-$k_y$ domain by applying a 3-dimensional Fourier transform. Next, a constant frequency slice is selected. This leaves a 2D matrix which captures the crossline and inline wavenumbers ($k_x$, $k_y$) as shown in Figure \ref{fig:FK-f_slice-data}. Note that the data map in a circle.

The lowest wavefield velocity, $v_{min}$, and the frequency, $f$, determine the maximum wavenumber, $k_{max}$;

\begin{equation}
	k_{max} = \frac{f}{v_{min}}.
	\label{eq:kmax}
\end{equation}

The total wavenumber, $k_T$ , must be smaller than the maximum wavenumber, $k_{max}$; 

\begin{equation}
	k_{T} = \sqrt{k_x^2 + k_y^2} < k_{max}.
	\label{eq:kT}
\end{equation}

Hence, the signal ”cone” is defined by a circle in the $k_x$-$k_y$ domain (see Figure \ref{fig:FK-f_slice-mask}). This is repeated for each frequency component such that the overall $f$-$k_x$-$k_y$ mask is a 3D cone (see Figure \ref{fig:FK-f_slice-data3d}). The cone can be sorted in a 2D-view as illustrated in Figure \ref{fig:FK-delphi-data} and Figure \ref{fig:FK-delphi-mask}.

\begin{figure}[h!]
	\centering
	\begin{subfigure}[t]{0.3\textwidth}
		\centering
		\includegraphics[width=\textwidth]{Plots/fkk/P_f_slice40}
		\caption{}
		\label{fig:FK-f_slice-data}
	\end{subfigure}
	%
	\centering
	\begin{subfigure}[t]{0.3\textwidth}
		\centering
		\includegraphics[width=\textwidth]{Plots/fkk/fkk-mask-slice40}
		\caption{}
		\label{fig:FK-f_slice-mask}
	\end{subfigure}
	%
	\centering
	\begin{subfigure}[t]{0.3\textwidth}
		\centering
		\includegraphics[width=\textwidth]{Plots/fkk/3dfk-cone}
		\caption{}
		\label{fig:FK-f_slice-data3d}
	\end{subfigure}
	
	\begin{subfigure}[t]{\textwidth}
		\centering
		\includegraphics[width=0.9\textwidth]{Plots/fkk/P_fkk_Delphi}
		\caption{}
		\label{fig:FK-delphi-data}
	\end{subfigure}
	\par\bigskip
	\begin{subfigure}[t]{\textwidth}
		\centering
		\includegraphics[width=0.9\textwidth]{Plots/fkk/fkk-mask-Delphi}
		\caption{}
		\label{fig:FK-delphi-mask}
	\end{subfigure}
	
	\caption{Illustration of the 3D $f$-$k_x$-$k_y$ filter. (a) is a \SI{40}{\hertz} frequency slice of the $f$-$k_x$-$k_y$ spectrum of the data in Figure \ref{fig:DataSorting}. (b) is a \SI{40}{\hertz} frequency slice of the $f$-$k_x$-$k_y$ mask where the white area equals one and the black area is zero. (c) shows the \SI{40}{\hertz} frequency slice of (a) sorted in a 3D cube. The red cone represents the edge of the 3D $f$-$k_x$-$k_y$ filter mask. (d) and (e) display the $f$-$k_x$-$k_y$ data spectrum and mask sorted in a 2D-view, i.e. each sub-cone refers to one inline wavenumber.}
	\label{fig:FKK-Mask}

\end{figure}


\subsection{Thresholding}

The blending noise is further attenuated by applying a sparsity constraint in the $x$-$t$ domain which is referred to as thresholding. Since the extension from 2D to 3D does not affect the thresholding I will explain it here. A description of thresholding can be found in \citet{Mahdad-Deblending-Method}.

\subsection{Iterations}

$f$-$k_x$-$k_y$ filtering and thresholding are applied iteratively to estimate and subtract the blending noise. Since this step works in the 3D case analogously to the 2D case presented by \citet{Mahdad-Deblending-Method} I will not describe it here.


\section{Example: Complex Synthetic Data}

Insert results of the Wide crossline source array
Show: Inline section of unblended, blended, pseudo-deblended and deblended data

\begin{figure}
	\centering
	\begin{subfigure}[t]{0.2\textwidth}
		\centering
		\includegraphics[width=\textwidth]{Plots/Seam/p_red_fil_x10}
		\caption{Unblended}
		\label{fig:Example-Unblendedx}
	\end{subfigure}
	%
	\centering
	\begin{subfigure}[t]{0.2\textwidth}
		\centering
		\includegraphics[width=\textwidth]{Plots/Seam/p_red_fil_x10_pseudo} % changed from x1 to x10
		\caption{Pseudo-deblended}
		\label{fig:Example-Pseudox}
	\end{subfigure}
	%
	\centering
	\begin{subfigure}[t]{0.2\textwidth}
		\centering
		\includegraphics[width=\textwidth]{Plots/Seam/p_red_fil_x10_debl} % changed from x1 to x10
		\caption{Deblended}
		\label{fig:Example-Deblendedx}
	\end{subfigure}
		
	\caption{\textbf{Wide Crossline Source Array}\\(a)-(c) show an inline slice of the unblended, pseudo-deblended, and deblended data respectively. The shown seismic sections are common-receiver gathers.}
	\label{fig:Example-Inline-Slices}

\end{figure}


\section{Conclusions}

This is the first sentence of the conclusions.

\section{Acknowledgements (Optional)}

This is the first sentence of the acknowledgements.

% \begin{thebibliography}{6pt}
%   \bibitem[{<reference>}]{<cite>} ...
% \end{thebibliography}
%
% or
%
% \bibliography{...}

% Biblio
\bibliographystyle{apalike}
\bibliography{my_bib} % in MyBib.bib you add all your reference information, following the correct format. Sometimes, the bib file needs to be built several times, as well as the main file, before all references occur correctly in your PDF. 


\end{document}