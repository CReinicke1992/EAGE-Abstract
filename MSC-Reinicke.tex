\documentclass{madrid15WS}

\usepackage{amssymb}
\usepackage{amsmath}
%\usepackage[FIGTOPCAP, hang, nooneline]{subfig}
\usepackage{subcaption}
\usepackage{placeins}
\usepackage{natbib}
\usepackage{siunitx}
%\usetikzlibrary{matrix}
\usepackage{todonotes}
\usepackage{caption}
\usepackage{verbatim}
\usepackage{rotating}


\begin{document}

\section{Introduction}

Blending is a recent seismic acquisition design, which allows seismic shots to interfere. I propose a new acquisition design based on blended crossline sources. 

test git

\begin{figure}[h!]
	\centering
	\begin{subfigure}[t]{0.3\textwidth}
		\centering
		\includegraphics[width = \textwidth]{Plots/Config-Conventional}
		\caption{}
		\label{fig:Intro-Config-Conventional}
	\end{subfigure}
	\qquad \qquad 
	\centering
	\begin{subfigure}[t]{0.3\textwidth}
		\centering
		\includegraphics[width = \textwidth]{Plots/Config-Xline-Blended}
		\caption{}
		\label{fig:Intro-Config-Xline-Blended}
	\end{subfigure}
	
	\caption{caption}
	\label{fig:Intro-Configs}
	
\end{figure}



Current processing techniques are not capable to deal with blended data. Consequently, the blended data must be deblended (separated) as if they were acquired in a conventional way.  


In contrast to existing blended-acquisition designs that only blend in 2D (inline direction and time), this design blends sources in 3D (inline direction, crossline direction and time). For this reason I propose a 3D deblending method. 

\section{Method and/or Theory}

This is the first sentence of the method and/or theory section.

\section{Examples (Optional)}

This is the first sentence of the example section.

\section{Results (Optional)}

This is the first sentence of the result section.

% \begin{figure}[!htb]
%   \centering
%   \includegraphics[width=0.6\textwidth]{....eps}
%   \caption{...}
% \end{figure}

\section{Conclusions}

This is the first sentence of the conclusions.

\section{Acknowledgements (Optional)}

This is the first sentence of the acknowledgements.

% \begin{thebibliography}{6pt}
%   \bibitem[{<reference>}]{<cite>} ...
% \end{thebibliography}
%
% or
%
% \bibliography{...}

\end{document}